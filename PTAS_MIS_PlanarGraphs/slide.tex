% Presentation Template: https://www.sharelatex.com/templates/presentations/conference-presentation

\documentclass{beamer}

\usepackage{xcolor}  
\usepackage{environ}

%
% Custom font for a frame.
%

\newcommand{\customframefont}[1]{
\setbeamertemplate{itemize/enumerate body begin}{#1}
\setbeamertemplate{itemize/enumerate subbody begin}{#1}
}

\NewEnviron{framefont}[1]{
\customframefont{#1} % for itemize/enumerate
{#1 % For the text outside itemize/enumerate
\BODY
}
\customframefont{\normalsize}
}

% There are many different themes available for Beamer. A comprehensive
% list with examples is given here:
% http://deic.uab.es/~iblanes/beamer_gallery/index_by_theme.html
% You can uncomment the themes below if you would like to use a different
% one:
%\usetheme{AnnArbor}
%\usetheme{Antibes}
%\usetheme{Bergen}
%\usetheme{Berkeley}
%\usetheme{Berlin}
%\usetheme{Boadilla}
%\usetheme{boxes}
%\usetheme{CambridgeUS}
%\usetheme{Copenhagen}
%\usetheme{Darmstadt}
%\usetheme{default}
%\usetheme{Frankfurt}
%\usetheme{Goettingen}
%\usetheme{Hannover}
%\usetheme{Ilmenau}
%\usetheme{JuanLesPins}
%\usetheme{Luebeck}
\usetheme{Madrid}
%\usetheme{Malmoe}
%\usetheme{Marburg}
%\usetheme{Montpellier}
%\usetheme{PaloAlto}
%\usetheme{Pittsburgh}
%\usetheme{Rochester}
%\usetheme{Singapore}
%\usetheme{Szeged}
%\usetheme{Warsaw}

\newcommand{\iph}[2]{
    \includegraphics[width=#1\textwidth,height=#1\textheight,keepaspectratio]{#2}
}
\newcommand{\ph}[1]{
    \includegraphics[width=0.5\textwidth,height=0.5\textheight,keepaspectratio]{#1}
}

\title{PTAS For MIS On Planar Graphs}

\author{Sourabh Aggarwal}

\institute[IIT Palakkad] % (optional, but mostly needed)
{
  Department of Computer Science And Engineering\\
  IIT Palakkad
}

\date{March 2019}

\subject{Theoretical Computer Science}

% Delete this, if you do not want the table of contents to pop up at
% the beginning of each subsection:
\AtBeginSubsection[]
{
  \begin{frame}<beamer>{Outline}
    \tableofcontents[currentsection,currentsubsection]
  \end{frame}
}

% Let's get started
\begin{document}

\begin{frame}
  \titlepage
\end{frame}

\begin{frame}{Outline}
  \tableofcontents
  % You might wish to add the option [pausesections]
\end{frame}

% Section and subsections will appear in the presentation overview
% and table of contents.
\section{Prerequisite}
\subsection{Recall}

\begin{frame}{\secname : \subsecname}
\begin{block}{Example}
    What is $|MIS|$ for $S_n$?

    \vspace{2\baselineskip}
    \ph{sn}
\end{block}
\end{frame}


\begin{frame}{\secname : \subsecname}
\begin{block}{MIS}
    What is $|MIS|$ for $S_n$? Ans. \textcolor{blue}{$n - 1$}

    \vspace{2\baselineskip}
    \ph{sn}

\end{block}
\end{frame}



\begin{frame}{\secname : \subsecname}
\begin{block}{Example}
    What is $|MIS|$ for $K_n$?
\end{block}
\end{frame}


\begin{frame}{\secname : \subsecname}
\begin{block}{MIS}
    What is $|MIS|$ for $K_n$? Ans. \textcolor{blue}{1}
\end{block}
\end{frame}


\begin{frame}{\secname : \subsecname}
\begin{block}{Euler's formula}
    If a finite, connected, planar graph is drawn in the plane without any edge intersections, and v is the number of vertices, e is the number of edges and f is the number of faces (regions bounded by edges, including the outer, infinitely large region), then
    $v - e + f = 2$
\end{block}
\end{frame}

\subsection{Outerplanar Graph}

\begin{frame}{Outerplanar Graphs}
\begin{block}{Definition}
    A graph is called outerplanar (1-outerplanar) if it has an embedding in the plane such that every vertex lies on the unbounded face.
\end{block}
\begin{block}{In Simple Terms}
    $\Leftrightarrow$ has a planar drawing for which all vertices belong to the outer face of the drawing.
\end{block}
% \begin{theorem}
% There are separate environments for theorems, examples, definitions and proofs.
% \end{theorem}
\end{frame}
\begin{frame}{Outerplanar Graphs}
\begin{example}
    \ph{o1}
\end{example}
\end{frame}

\begin{frame}{Outerplanar Graphs}
\begin{example}
    \ph{o2}
\end{example}
\end{frame}

\begin{frame}{Outerplanar Graphs}
\begin{example}
    Is this outerplanar?

    \ph{o3}


\end{example}
\end{frame}
    
\begin{frame}{Outerplanar Graphs}
\begin{example}
    Yes, as it is isomorphic to the following graph.

    \ph{o4}


\end{example}
\end{frame}


\begin{frame}{Outerplanar Graphs}
\begin{example}
    Is this outerplanar?

    \ph{o5}


\end{example}
\end{frame}


\begin{frame}{Outerplanar Graphs}
\begin{example}
    No, but it is planar.

    \ph{o6}


\end{example}
\end{frame}

\begin{frame}{Outerplanar Graphs}
\begin{block}{Definition}
    Given a planar embedding E of a planar graph G, we
    define k-level nodes according to the faces of the embedding. Every node on the exterior face is called an exterior
    or level 1 node. Exterior edges are defined analogously. Call a cycle of
    level $i$ nodes a $level$ $i$ $face$ if it is an interior face in the subgraph induced by
    the level $i$ nodes. For each $level$ $i$ $face$ $f$, let $G_f$ be the subgraph induced by all
    nodes placed inside $f$ in this embedding. Then the nodes on the exterior face
    of $G_f$ are at level $i + 1$. 
\end{block}
\end{frame}


\begin{frame}{Outerplanar Graphs}
\begin{block}{Definition}
    Basically, in each
    step, starting at 1, remove all nodes on the exterior face.
    Repeat until there are no nodes left. The level of a node
    is the step in which it was removed. Using an appropriate data structure for the planar embedding, such as the
    one used by Lipton and Tarjan [Lipton and Tarjan,
    1979], this can be done in linear time.
    A planar embedding is called k-outerplanar if there are
    no nodes of level $>$ k, which means the algorithm terminates after at most k steps. A planar graph is called
    k-outerplanar if it has a k-outerplanar embedding.  
\end{block}
\end{frame}
\begin{frame}
    \begin{example}
        \iph{0.75}{o7}
    \end{example}
\end{frame}
\subsection{Fixed Parameter Tractability}
\begin{frame}
    \begin{block}{Definition}
        A problem is considered fixed parameter tractable if an algorithm exists which solves the problem in running time $f(k) * n^{O(1)}$ where $k$ is an additional parameter depending upon input and n is the size of the input (in our case we will treat is as number of nodes) and $f$ is an arbitrary function depending only on $k$.
    \end{block}

\end{frame}
\section{High Level Plan}
\begin{frame}{\secname}
    We will use this new notion of $k$-outerplanarity to construct a polynomial time approximation scheme for maximum independent set on planar graphs, which can be
easily adapted to other NP-complete problems. The goal
is an algorithm that, for a freely chosen but fixed $k$, solves maximum independent set with an approximation ratio
of $k / (k + 1)$ and a linear complexity with respect to the
number of nodes n.
Basically, we want to exactly solve maximum independent on disjunct subgraphs that are $k$-outerplanar and
ignore some nodes in the graph to merge the subgraphs
without violating the conditions of the maximum independent set.
\newline
\newline
\newline
\textbf{PS}: Approximation Ratio? For maximization problems it is atleast $(1 - \epsilon)$ therefore fix $k = (1/\epsilon) - 1$ (do math).
\end{frame}
\section{Bakers Algorithm}
\subsection{Theorem 1}
\begin{frame}{\secname : \subsecname}
   \begin{theorem} 
    [Baker, 1994] Let $k$ be a positive integer. Given a $k$-outerplanar embedding of a $k$-outerplanar
    graph G, an optimal solution for maximum independent
    set can be obtained in time $O(8^kn)$, where $n$ is the number of nodes.
    \end{theorem}

    Its proof will be done later.
\end{frame}
\subsection{Main Algorithm}
\begin{frame}{\secname : \subsecname}
    \begin{enumerate}
        \item Generate a planar embedding of G using the linear time algorithm of Hopcroft and Tarjan [Hopcroft
        and Tarjan, 1974] and compute the level of every
        node.
        \item For each $i$; $0 \leq i \leq k$, do the following:
        \begin{enumerate}
        \item Remove all nodes where the nodes level
        mod (k + 1) equals i, splitting the graph into
        components with an (already computed) k-outerplanar embedding.
        \item Use Theorem 1 to exactly solve maximum independent set on all those components and
        take the union of the solutions. This is possible as the components are disjunct and no
        edges can exist between two components.
        \end{enumerate}
        \item Take the best solution for all $i$ as the solution for
        the entire graph.
    \end{enumerate}
\end{frame}
\subsection{Analysis of the given Algorithm}
\begin{frame}{\secname : \subsecname}
    To see that this gives a solution with approximation ratio
$k/(k + 1)$, consider the optimal solution $S_{OPT}$ . For some $i$
as defined above, at most $1/(k + 1)$ of the nodes in $S_{OPT}$
are at a level congruent to $i$ 

[As suppose this was not the case $\rightarrow S_{OPT} > (k + 1) * (1/(k + 1) * S_{OPT})$ i.e. $S_{OPT} > S_{OPT}$ which is absurd]. 

As the other components
are solved exactly, the union of their solutions has at least
the size of $S_{OPT} - 1/(k + 1) * S_{OPT}$ i.e. $k/(k + 1) * S_{OPT}$ giving our desired approximation factor. 

[As $Rem \leq S_{OPT}/(k + 1) \rightarrow -Rem \geq -S_{OPT}/(k + 1) \rightarrow S_{OPT} - Rem \geq k/(k+1)*S_{OPT}$]

This leads to the second theorem:
\end{frame}
\subsection{Theorem 2}
\begin{frame}{\secname : \subsecname}
    \begin{theorem}
        [Baker, 1994] For fixed $k$, there is an
$O(8^kkn)$-time algorithm for maximum independent set
that achieves a solution of size at least $k/k+1$ optimal for
general planar graphs. 
    \end{theorem}

    We have proved this [assuming theorem 1]. So, what remains is to prove theorem 1
\end{frame}
\subsection{Linear Time Algorithm for Outerplanar Graphs}
\begin{frame}{\secname : \subsecname}
    \begin{block}{Given a connected outerplanar graph, do the following}
    \begin{itemize}
        \item Replace each bridge (x, y) by two
        edges between x and y. This will allow us to treat a bridge as a face rather
        than as a special case. Call the resulting graph $G$.

        \item The maximum independent set for $G$ will be computed by recursively
processing an ordered, rooted tree $\bar{G}$ that represents the structure of G. Each
leaf of $\bar{G}$ will represent an exterior edge of $G$ and every other vertex will represent
a face of $G$ and will be called a $face$ vertex. $\bar{G}$ is constructed as follows:
        \begin{enumerate}
        \item Construct a graph vertex for every interior face and
        every exterior edge.

        \item Draw edges between every edge vertex and the vertex of the face the edge is contained in, and between
        every two vertices of faces that share an edge.

        \item Need to handle cut points (presented later)      
        \end{enumerate}
    \end{itemize}
    \end{block}

\end{frame}



\begin{frame}{\secname : \subsecname}
        \iph{0.9}{lin1}

\end{frame}

\begin{frame}{\secname : \subsecname}
    \begin{block}{Question}
        Why are we getting a Tree? 
    \end{block}

\end{frame}

\begin{frame}{\secname : \subsecname}
    \begin{block}{Question}
        Why are we getting a Tree? Will we always get a tree for a planar embedding?
    \end{block}

\end{frame}

\begin{frame}{\secname : \subsecname}
    \begin{block}{Question}
        Why are we getting a Tree? Will we always get a tree for a planar embedding - Ans. \textcolor{blue}{No, consider planar embedding of $K_4$. So, this property is actually the result of outerplanarity condition.}
    \end{block}

\end{frame}



\begin{framefont}{\small}
\begin{frame}{\secname : \subsecname}
    \begin{block}{Labelling our tree}
        \begin{enumerate}
        \item The planar embedding induces a cyclic ordering on the
        edges of each vertex in the tree [for our discussion, we will follow counterclockwise orientation]. Choosing a face vertex $v$ as the root and
        choosing which child of $v$ is to be its leftmost child determine the parent and
        ordering of children for every other vertex of G [We will see this in diagram].
        \item Label the vertices of $\bar{G}$ recursively, as follows: Label each leaf of the tree
        with the oriented exterior edge it represents. Label each face vertex with the
        first and last nodes in its children’s labels.
        \item If a face vertex is labeled (x, y), the leaves of its subtree represent a directed
        walk of exterior edges in a counterclockwise direction from x to y. For the
        root, x = y and the directed walk covers all the exterior edges. For any other
        face vertex $v$, x $\neq$ y, and (x, y) is an interior edge shared by the face
        represented by $v$ and the face represented by its parent in the tree.
        \end{enumerate}
    \end{block}

\end{frame}
\end{framefont}


\begin{frame}{\secname : \subsecname}
    \iph{0.9}{lin2}

\end{frame}

\begin{frame}{\secname : \subsecname}
    \begin{block}{Handling Cutpoints}
        Cutpoints are nodes,
whose removal disconnects the graph. As we removed bridges, a cutpoint
is a node where at least two faces meet without
sharing an edge, therefore disconnecting the tree for
G. Draw an edge between two vertices of faces that
both contain the cutpoint and are part of different
components until G is connected.
    \end{block}
\end{frame}

\begin{frame}{\secname : \subsecname}
    \iph{0.9}{lin3}

\end{frame}

\begin{frame}{\secname : \subsecname}
    \begin{block}{Labelling our tree}
        Labelling is done same as before. Note that now it is possible for a face vertex labelled (x, y) and x = y in which case, the label
        represents a cutpoint shared by two faces, as for the vertex labeled (9,9) in
        below figure.
    \end{block}

\end{frame}

\begin{frame}{\secname : \subsecname}
    \iph{0.9}{lin4}

\end{frame}

\end{document}


