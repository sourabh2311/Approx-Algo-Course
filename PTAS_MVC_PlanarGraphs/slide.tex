% Presentation Template: https://www.sharelatex.com/templates/presentations/conference-presentation


\documentclass{beamer}

% There are many different themes available for Beamer. A comprehensive
% list with examples is given here:
% http://deic.uab.es/~iblanes/beamer_gallery/index_by_theme.html
% You can uncomment the themes below if you would like to use a different
% one:
%\usetheme{AnnArbor}
%\usetheme{Antibes}
%\usetheme{Bergen}
%\usetheme{Berkeley}
%\usetheme{Berlin}
%\usetheme{Boadilla}
%\usetheme{boxes}
%\usetheme{CambridgeUS}
%\usetheme{Copenhagen}
%\usetheme{Darmstadt}
%\usetheme{default}
%\usetheme{Frankfurt}
%\usetheme{Goettingen}
%\usetheme{Hannover}
%\usetheme{Ilmenau}
%\usetheme{JuanLesPins}
%\usetheme{Luebeck}
\usetheme{Madrid}
%\usetheme{Malmoe}
%\usetheme{Marburg}
%\usetheme{Montpellier}
%\usetheme{PaloAlto}
%\usetheme{Pittsburgh}
%\usetheme{Rochester}
%\usetheme{Singapore}
%\usetheme{Szeged}
%\usetheme{Warsaw}

\newcommand{\iph}[2]{
    \includegraphics[width=#1\textwidth,height=#1\textheight,keepaspectratio]{#2}
}
\newcommand{\ph}[1]{
    \includegraphics[width=0.5\textwidth,height=0.5\textheight,keepaspectratio]{#1}
}

\title{PTAS For MVC On Planar Graphs}

\author{Sourabh Aggarwal}

\institute[IIT Palakkad] % (optional, but mostly needed)
{
  Department of Computer Science And Engineering\\
  IIT Palakkad
}

\date{March 2019}

\subject{Theoretical Computer Science}

% Delete this, if you do not want the table of contents to pop up at
% the beginning of each subsection:
\AtBeginSubsection[]
{
  \begin{frame}<beamer>{Outline}
    \tableofcontents[currentsection,currentsubsection]
  \end{frame}
}

% Let's get started
\begin{document}

\begin{frame}
  \titlepage
\end{frame}

\begin{frame}{Outline}
  \tableofcontents
  % You might wish to add the option [pausesections]
\end{frame}

% Section and subsections will appear in the presentation overview
% and table of contents.
\section{Prerequisite}

\subsection{Outerplanar Graph}

\begin{frame}{Outerplanar Graphs}
\begin{block}{Definition}
    A graph is called outerplanar (1-outerplanar) if it has an embedding in the plane such that every vertex lies on the unbounded face.
\end{block}
\begin{block}{In Simple Terms}
    $\Leftrightarrow$ has a planar drawing for which all vertices belong to the outer face of the drawing.
\end{block}
% \begin{theorem}
% There are separate environments for theorems, examples, definitions and proofs.
% \end{theorem}
\end{frame}
\begin{frame}{Outerplanar Graphs}
\begin{example}
    \ph{o1}
\end{example}
\end{frame}

\begin{frame}{Outerplanar Graphs}
\begin{example}
    \ph{o2}
\end{example}
\end{frame}

\begin{frame}{Outerplanar Graphs}
\begin{example}
    Is this outerplanar?

    \ph{o3}


\end{example}
\end{frame}
    
\begin{frame}{Outerplanar Graphs}
\begin{example}
    Yes, as it is isomorphic to the following graph.

    \ph{o4}


\end{example}
\end{frame}


\begin{frame}{Outerplanar Graphs}
\begin{example}
    Is this outerplanar?

    \ph{o5}


\end{example}
\end{frame}


\begin{frame}{Outerplanar Graphs}
\begin{example}
    No, but it is planar.

    \ph{o6}


\end{example}
\end{frame}

\begin{frame}{Outerplanar Graphs}
\begin{block}{Definition}
    Given a planar embedding E of a planar graph G, we
    define k-level nodes according to the faces of the embedding. Every node on the exterior face is called an exterior
    or level 1 node. Exterior edges are defined analogously. In each
    step, starting at 1, remove all nodes on the exterior face.
    Repeat until there are no nodes left. The level of a node
    is the step in which it was removed. Using an appropriate data structure for the planar embedding, such as the
    one used by Lipton and Tarjan [Lipton and Tarjan,
    1979], this can be done in linear time.
    A planar embedding is called k-outerplanar if there are
    no nodes of level $>$ k, which means the algorithm terminates after at most k steps. A planar graph is called
    k-outerplanar if it has a k-outerplanar embedding. 
\end{block}
\end{frame}
\begin{frame}
    \begin{example}
        \iph{0.75}{o7}
    \end{example}
\end{frame}
\subsection{Fixed Parameter Tractability}
\begin{frame}
    \begin{block}{Definition}
        A problem is considered fixed parameter tractable if an algorithm exists which solves the problem in running time $f(k) * n^{O(1)}$ where $k$ is an additional parameter depending upon input and n is the size of the input (in our case we will treat is as number of nodes) and $f$ is an arbitrary function depending only on $k$.
    \end{block}

\end{frame}
\section{High Level Plan}
\begin{frame}{\secname}
    We will use this new notion of $k$-outerplanarity to construct a polynomial time approximation scheme for maximum independent set on planar graphs, which can be
easily adapted to other NP-complete problems. The goal
is an algorithm that, for a freely chosen but fixed $k$, solves maximum independent set with an approximation ratio
of $k / (k + 1)$ and a linear complexity with respect to the
number of nodes n.
Basically, we want to exactly solve maximum independent on disjunct subgraphs that are $k$-outerplanar and
ignore some nodes in the graph to merge the subgraphs
without violating the conditions of the maximum independent set.
\newline
\newline
\newline
\textbf{PS}: Approximation Ratio? For maximization problems it is atmost $(1 - \epsilon)$ therefore fix $k = (1/\epsilon) - 1$ (do math).
\end{frame}
\section{Bakers Algorithm}
\subsection{Theorem 1}
\begin{frame}{\secname : \subsecname}
   \begin{theorem} 
    [Baker, 1994] Let $k$ be a positive integer. Given a $k$-outerplanar embedding of a $k$-outerplanar
    graph G, an optimal solution for maximum independent
    set can be obtained in time $O(8^kn)$, where $n$ is the number of nodes.
    \end{theorem}

    Its proof will be done later.
\end{frame}
\subsection{Main Algorithm}
\begin{frame}{\secname : \subsecname}
    \begin{enumerate}
        \item Generate a planar embedding of G using the linear time algorithm of Hopcroft and Tarjan [Hopcroft
        and Tarjan, 1974] and compute the level of every
        node.
        \item For each $i$; $0 \leq i \leq k$, do the following:
        \begin{enumerate}
        \item Remove all nodes where the nodes level
        mod (k + 1) equals i, splitting the graph into
        components with an (already computed) k-outerplanar embedding.
        \item Use Theorem 1 to exactly solve maximum independent set on all those components and
        take the union of the solutions. This is possible as the components are disjunct and no
        edges can exist between two components.
        \end{enumerate}
        \item Take the best solution for all $i$ as the solution for
        the entire graph.
    \end{enumerate}
\end{frame}
\subsection{Analysis of the given Algorithm}
\begin{frame}{\secname : \subsecname}
    To see that this gives a solution with approximation ratio
$k/(k + 1)$, consider the optimal solution $S_{OPT}$ . For some $i$
as defined above, at most $1/(k + 1)$ of the nodes in $S_{OPT}$
are at a level congruent to $i$ 

[As suppose this was not the case $\rightarrow S_{OPT} > (k + 1) * (1/(k + 1) * S_{OPT})$ i.e. $S_{OPT} > S_{OPT}$ which is absurd]. 

As the other components
are solved exactly, the union of their solutions has at least
the size of $S_{OPT} - 1/(k + 1) * S_{OPT}$ i.e. $k/(k + 1) * S_{OPT}$ giving our desired approximation factor. 

[As $Rem \leq S_{OPT}/(k + 1) \rightarrow -Rem \geq -S_{OPT}/(k + 1) \rightarrow S_{OPT} - Rem \geq k/(k+1)*S_{OPT}$]

This leads to the second theorem:
\end{frame}
\subsection{Theorem 2}
\begin{frame}{\secname : \subsecname}
    \begin{theorem}
        [Baker, 1994] For fixed $k$, there is an
$O(8^kkn)$-time algorithm for maximum independent set
that achieves a solution of size at least $k/k+1$ optimal for
general planar graphs. 
    \end{theorem}

    We have proved this [assuming theorem 1]. So, what remains is to prove theorem 1
\end{frame}
\end{document}


